\section{NFT}
\subsection{What is NFT}
\begin{itemize}
  \item Fungible means ersetztbar.
  \item Unique token on a public blockchain
  \item Guarantees that a digital asset is unique and not interchangeable
  \item Can be any digital data that can be hashed (only hash is stored on-chain)
  \item With NFT: proof of ownership (you can copy the digital data, but the ownership remains)
\end{itemize}

\subsection{Difference fungible/non-fugible}
Fungible assets are the assets that you can swap with another similar entity. 
For example, you can swap currency or shares with similar values. 
You can exchange a one-dollar bill with any other one-dollar bill as all of them represent same value.

On the other hand, non-fungible assets are the opposite and cannot be swapped for one another.
For example, a house could be easily considered a non-fungible asset as it would have some unique properties.
When it comes to the crypto world, representation of assets in the digital form would definitely have to consider the aspects of fungibility and non-fungibility.

\subsubsection{Examples}
Popular NFT collection is CryptoPunks.
Owner is stored in the Ethereum blockchain and can be traded decentralized.

\begin{itemize}
  \item Collectible media (football, basketball players)
  \item Jack Dorsey sold first twitter post
  \item Tickets
  \item NFT items in games, e.g., CS:GO skins
  \item Artists sell music as NFT
\end{itemize}

\subsection{Fan Token}
Fan Tokens are not NFTs!\\
They are Fungible Tokens, so replaceable.

